\documentclass{assignment}
\ProjectInfos{量子信息导论}{PHYS5251P}{2021-2022 学年第一学期}{第七次作业(量子计算第三章)}{截止日期:2022. 1. 3(周三)}{陈稼霖}[https://github.com/Chen-Jialin]{SA21038052}

\begin{document}
\begin{prob}
    证明两种图态定义的等价性.
\end{prob}
\begin{pf}
    要证两种图态定义的等价性, 即证显式定义的图态是隐式定义中算符 $K_p$ ($p\in V$) 的本征值为 $1$ 的本征态.\\
    显示定义的图态的表达式为
    \begin{align}
        \lvert G\rangle=S_C\lvert+\rangle_1\cdots\lvert+\rangle_n=\prod_{\{a,b\}\in E}CZ_{ab}\lvert+\rangle_1\cdots\lvert+\rangle_n.
    \end{align}
    隐式定义中算符
    \begin{align}
        K_p=\sigma_p^x\prod_{b\in N_p}\sigma_b^z.
    \end{align}
    将算符 $K_p$ 作用于图态 $\lvert G\rangle$ 上有
    \begin{align}
        K_p\lvert G\rangle=K_pS_C\lvert+\rangle_1\cdots\lvert+\rangle_n=K_p\prod_{\{a,b\}}CZ_{ab}\lvert+\rangle_1\cdots\lvert+\rangle_n.
    \end{align}
    由于 $CZ_{ab}$ 和 $CZ_{\mu\nu}$ 之间是可对易的且 $CZ_{ab}=\lvert 0\rangle_a\langle 0\rvert\otimes I_b+\lvert 1\rangle_a\langle 1\rvert\otimes\sigma_b^z\Longrightarrow S_C^2=I$, 从而
    \begin{align}
        K_p\lvert G\rangle=IK_p\lvert G\rangle=S_CS_CK_p\prod_{\{a,b\}\in E}CZ_{ab}\lvert+\rangle_1\cdots\lvert+\rangle_n=S_C\prod_{\{\mu,\nu\}}CZ_{\mu\nu}K_p\prod_{\{a,b\}\in E}CZ_{ab}\lvert+\rangle_1\cdots\lvert+\rangle_n.
    \end{align}
    利用算符 $CZ_{ab}$ 与 Pauli 算符之间的对易关系:
    \begin{align}
        CZ_{ab}\sigma_a^xCZ_{ab}=&\sigma_a^x\otimes\sigma_b^z,\\
        CZ_{ab}\sigma_b^xCZ_{ab}=&\sigma_a^z\otimes\sigma_b^x,\\
        CZ_{ab}\sigma_c^xCZ_{ab}=&\sigma_c^x\quad(c\neq a,b),\\
        CZ_{ab}\sigma_i^cCZ_{ab}=&\sigma_i^z\quad\forall i,
    \end{align}
    无妨将 $p$ 视为某个控制比特, 则有 $CZ_{pb}\sigma_p^xCZ_{pb}=\sigma_p^x\otimes\sigma_b^z$, 得到 $\sigma_b^z$ 与 $K_p$ 中的 $\sigma_b^z$ 相消, 最终得到
    \begin{align}
        K_p\lvert G\rangle=S_C\sigma_p^x\lvert+\rangle_1\cdots\lvert+\rangle_n=S_C\lvert+\rangle_1\cdots\lvert+\rangle_n=\lvert G\rangle,
    \end{align}
    故显式定义的图态 $\lvert G\rangle$ 是隐式定义中算符 $K_p$ ($p\in V$) 的本征值为 $1$ 的本征态, 显式定义和隐式定义等价.
\end{pf}

\begin{prob}
    证明 11 页中 LC 变换后图态的形式.
\end{prob}
\begin{pf}
    要证 $U_a^{\tau}\lvert G\rangle=e^{-i\frac{\pi}{4}\sigma_a^x}\prod_{k\in N_a}e^{i\frac{\pi}{4}\sigma_k^z}\lvert G\rangle$ 是图 $\tau_a(G)$ 的图态, 即证 $U_a^{\tau}\lvert G\rangle=e^{-i\frac{\pi}{4}\sigma_a^x}\prod_{k\in N_z}e^{i\frac{\pi}{4}\sigma_k^z}\lvert G\rangle$ 是图 $\tau_a(G)$ 对应的稳定子群 $S$ 的本征值为 $1$ 的共同本征态. 设图 $G$ 中点 $p$ 对应的稳定子为 $K_p$. 简记 $U_a^{\tau}$ 为 $U_a$. 由于
    \begin{align}
        U_a\lvert G\rangle=U_aK_p\lvert G\rangle=U_aK_pU_a^{\dagger}U_a\lvert G\rangle=U_aK_pU_a^{\dagger}\lvert\tau_a(G)\rangle,
    \end{align}
    因此只需证 $U_aK_pU_a^{\dagger}$ 是图 $\tau_a(G)$ 的稳定子.

    \begin{itemize}
        \item[(1)] 当 $p=a$ 时, 有
        \begin{align}
            U_aK_aU_a^{\dagger}=\left(e^{-i\frac{\pi}{4}\sigma_a^x}\prod_{k\in N_a}e^{i\frac{\pi}{4}\sigma_k^z}\right)\left(\sigma_a^x\prod_{j\in N_a}\sigma_j^z\right)\left(e^{-i\frac{\pi}{4}\sigma_a^x}\prod_{l\in N_a}e^{i\frac{\pi}{4}\sigma_l^z}\right)^{\dagger}.
        \end{align}
        利用 $e^{-i\frac{\pi}{4}\sigma_a^x}=\frac{1}{\sqrt{2}}\begin{pmatrix}
            1&1\\
            1&-1
        \end{pmatrix}\begin{pmatrix}
            e^{-i\frac{\pi}{4}}&0\\
            0&e^{i\frac{\pi}{4}}
        \end{pmatrix}\frac{1}{\sqrt{2}}\begin{pmatrix}
            1&1\\
            1&-1
        \end{pmatrix}=\frac{1}{\sqrt{2}}\begin{pmatrix}
            1&1\\
            1&-1
        \end{pmatrix}\begin{pmatrix}
            \sqrt{-i}&0\\
            0&\sqrt{i}
        \end{pmatrix}\frac{1}{\sqrt{2}}\begin{pmatrix}
            1&1\\
            1&-1
        \end{pmatrix}=\sqrt{-i\sigma_a^x}$, $e^{i\frac{\pi}{4}\sigma_k^z}=\begin{pmatrix}
            e^{i\frac{\pi}{4}}&0\\
            0&e^{-i\frac{\pi}{4}}
        \end{pmatrix}=\begin{pmatrix}
            \sqrt{i}&0\\
            0&\sqrt{-i}
        \end{pmatrix}=\sqrt{i\sigma_k^z}$ 得
        \begin{align}
            U_aK_aU_a^{\dagger}=\sqrt{-i\sigma_a^x}\sigma_a^x\sqrt{i\sigma_a^x}\prod_{b\in N_a}\sqrt{i\sigma_b^z}\sigma_b^z\sqrt{-i\sigma_b^z}=\sigma_a^x\prod_{b\in N_a}\sigma_b^z=K_a,
        \end{align}
        局域操作 $\tau_a$ 不改变顶点 $a$ 与 $N_a$ 之间的连接关系, 故 $U_aK_aU_a^{\dagger}$ 是图 $\tau_a(G)$ 的稳定子.
        \item[(2)] 当 $p\in N_a$ 时, 算符 $K_p$ 与 $U_a$ 的支集 $\sup(K_a)$ 和 $\sup(U_a)$ 的交集为
        \begin{align}
            \sup(K_p)\cap\sup(U_a)=(p\cup N_p)\cap(a\cup N_a)=p\cup a\cup(N_p\cap N_a),
        \end{align}
        且图 $\tau_a(G)$ 在 $p$ 处的稳定子为
        \begin{align}
            K_p'=K_p\prod_{b\in N_a-p}\sigma_b^z=\sigma_p^x\prod_{b\in N_p}\sigma_b^z\prod_{b\in N_a-p}\sigma_b^z=\sigma_p^x\prod_{b\in N_p\cup(N_a-p)-N_p\cap N_a}\sigma_b^z,
        \end{align}
        因此
        \begin{align}
            \notag U_aK_pU_a^{\dagger}=&\left(\sqrt{-i\sigma_a^x}\prod_{k\in N_a}\sqrt{i\sigma_k^z}\right)\left(\sigma_p^x\prod_{j\in N_p}\sigma_j^z\right)\left(\sqrt{i\sigma_a^x}\prod_{l\in N_a}\sqrt{-i\sigma_l^z}\right)\\
            \notag=&\sqrt{-i\sigma_a^x}\sigma_a^z\sqrt{i\sigma_a^x}\,\sqrt{i\sigma_p^z}\sigma_p^z\sqrt{-i\sigma_p^z}\prod_{b\in N_p\cap N_a}\sqrt{i\sigma_b^z}\sigma_b^z\sqrt{-i\sigma_b^z}\prod_{c\in N_p-(N_p\cap N_a)-a}\sigma_c^z\\
            \notag=&(-\sigma_p^y)(-\sigma_a^y)\prod_{b\in N_p-a}\sigma_b^z\\
            \notag=&\sigma_p^y\sigma_a^y\sigma_a^z\prod_{b\in N_p}\sigma_b^z\\
            \notag=&\sigma_p^y\sigma_p^z\sigma_a^y\sigma_a^z\prod_{b\in N_a}\sigma_b^z\prod_{c\in N_p\cup(N_a-p)-N_p\cap N_a}\sigma_c^z\\
            \notag=&\sigma_a^x\prod_{b\in N_a}\sigma_b^z\sigma_p^x\prod_{c\in N_p\cup(N_a-p)-N_p\cap N_a}\sigma_c^z\\
            \notag=&K_aK_p',
        \end{align}
        从而
        \begin{align}
            U_aK_pU_a^{\dagger}\lvert\tau_a(G)\rangle=K_aK_p'\lvert\tau_a(G)\rangle=K_a\lvert\tau_a(G)\rangle,
        \end{align}
        由于局域操作 $\tau_a$ 不改变顶点 $a$ 与 $N_a$ 之间的连接关系, 故 $K_a$ 是图 $\tau_a(G)$ 的稳定子,
        \begin{align}
            U_aK_pU_a^{\dagger}\lvert\tau_a(G)\rangle=\lvert\tau_a(G)\rangle,
        \end{align}
        进而 $U_aK_pU_a^{\dagger}$ 是图 $\tau_a(G)$ 的稳定子.
        \item[(3)] 当 $p\neq a$ 且 $p\notin N_a$ 时, $K_p$ 与 $U_a$ 的支集 $\sup(K_p)$ 与 $\sup(U_a)$ 之间的交集为 $N_a\cap N_p$, 图 $\tau_a(G)$ 对应的稳定子为 $K_p'=K_p$, 从而
        \begin{align}
            \notag U_aK_pU_a^{\dagger}=&\left(\sqrt{-i\sigma_a^x}\prod_{k\in N_a}\sqrt{i\sigma_k^z}\right)\left(\sigma_p^x\prod_{j\in N_p}\sigma_j^z\right)\left(\sqrt{i\sigma_a^x}\prod_{l\in N_a}\sqrt{-i\sigma_l^z}\right)\\
            \notag=&\prod_{b\in N_a\cap N_p}\sqrt{i\sigma_b^z}\sigma_p^x\prod_{c\in N_p}\sigma_c^z\prod_{d\in N_a\cap N_p}\sqrt{-i\sigma_d^z}\\
            \notag=&\sigma_p^x\prod_{c\in N_p}\sigma_c^z\\
            =&K_p=K_p',
        \end{align}
        进而
        \begin{align}
            U_aK_pU_a^{\dagger}\lvert\tau_a(G)\rangle=K_p'\lvert\tau_a(G)\rangle=\lvert\tau_a(G)\rangle,
        \end{align}
        故 $U_aK_pU_a^{\dagger}$ 是图 $\tau_a(G)$ 的稳定子.
    \end{itemize}

    综上, $\lvert\tau_a(G)\rangle$ 是图 $\tau_a(G)$ 的稳定子群的本征值为 $1$ 的共同本征态, 故量子态 $\lvert\tau_a(G)\rangle$ 是图 $\tau_a(G)$ 对应的图态.
\end{pf}

\begin{prob}
    推导第 12 页中的 $P_x$ 测量的结果.
\end{prob}
\begin{pf}
    由于 $\lvert G\rangle$ 为 $G$ 对应的图态, 故 $\langle G\rvert K_p\lvert G\rangle=\langle G\vert G\rangle=1$, ($p\in V$).

    稳定子算符集合 $S_a=\{K_p\mid p\in V-a\}$ 可分为两部分:
    \begin{itemize}
        \item[(1)] $S_a^1$ 中算符的支集不含顶点 $a$;
        \item[(2)] $S_a^2$ 中算符的支集含顶点 $a$.
    \end{itemize}
    若 $a\in S_a^1$, 即 $K_p$ 的支集中不包含 $a$, 则 $K_p$ 与 $a$ 上的测量操作 $P_a^{z,\pm}=\frac{1}{2}(I\pm\sigma_a^z)$ 可对易, 因此测量操作 $P_a$ 不影响 $K_p$ 的测量值, 即 $\langle G_a\rvert K_p\lvert G_a\rangle=\langle G\rvert K_p\lvert G\rangle=1$, 其中 $\lvert G_a\rangle$ 表示测量操作 $P_a$ 后 (去除测量比特 $a$) 系统的量子态.

    若 $p\in S_a^2$, 即 $K_p$ 的支集中包含 $a$, $p\in N_a$, 即 $K_p$ 作用在 $a$ 上的算符为 $\sigma_a^z$. 此时,
    \begin{itemize}
        \item 若 $\sigma_a^z$ 的测量结果为 $1$, 则
        \begin{align}
            \langle G_a\rvert K_p'\lvert G_a\rangle=\langle G\rvert K_p'\sigma_a^z\lvert G\rangle=\langle G\rvert K_p\lvert G\rangle=1,
        \end{align}
        \item 若 $\sigma_a^z$ 的测量结果为 $-1$, 则
        \begin{align}
            \langle G_a\rvert K_p'\lvert G_a\rangle=-1.
        \end{align}
    \end{itemize}
    由此, 测量后的量子态 $\lvert G_a\rangle$ 是稳定算符 $S_1$ 和 $S_2'$ (由 $K_p'$ 组成) 的共同本征态:
    \begin{align}
        \langle G_a\rvert K_p'\lvert G_a\rangle=&1\quad(p\in N_a,\text{ 测量结果为 }1),\\
        \langle G_a\rvert K_p'\lvert G_a\rangle=&-1\quad(p\in N_a,\text{ 测量结果为 }-1),\\
        \langle G_a\rvert K_p\lvert G_a\rangle=&1\quad(p\notin N_a).
    \end{align}
    当比特 $a$ 上 $\sigma^z$ 的测量为 $-1$ 时, 对 $N_a$ 上的量子比特做 $\sigma^z$ 操作就可将测量后的量子态 $\lvert G_a\rangle$ 转化为标准的图态 $\lvert G-a\rangle$:
    \begin{align}
        \notag\langle G_a\rvert\prod_{b\in N_a}\sigma_b^zK_p\prod_{c\in N_a}\sigma_c^z\lvert G_a\rangle=&\langle G_a\rvert\prod_{b\in N_a}\sigma_b^z\sigma_p^x\prod_{d\in N_p}\sigma_d^z\prod_{c\in N_a}\sigma_c^z\lvert G_a\rangle=\langle G_a\rvert(-\sigma_p^x)\prod_{d\in N_p}\sigma_d^z\lvert G_a\rangle=\langle G_a\rvert(-K_p)\lvert G_a\rangle\\
        =&1\quad(p\in N_a,\text{ 测量结果为 }-1).
    \end{align}
    因此
    \begin{align}
        P_a^{z,\pm}\lvert G\rangle=\frac{1}{\sqrt{2}}\lvert z,\pm\rangle_a\otimes U_a^{z,\pm}\lvert G-a\rangle,
    \end{align}
    其中 $U_a^{z,+}=1$, $U_a^{z,-}=\prod_{k\in N_a}\sigma_k^z$.

    下证 $P_a^{y,\pm}\lvert G\rangle=\frac{1}{\sqrt{2}}\lvert y,\pm\rangle_a\otimes U_a^{y,\pm}\lvert\tau_a(G)-a\rangle$: 利用
    \begin{align}
        \notag(U_a^{\tau}(G))^{\dagger}P_a^{z,\pm}U_a^{\tau}(G)=&\left(e^{-i\frac{\pi}{4}\sigma_a^x}\prod_{k\in N_a}e^{i\frac{\pi}{4}\sigma_k^z}\right)^{\dagger}P_a^{z,\pm}\left(e^{-i\frac{\pi}{4}\sigma_a^x}\prod_{k\in N_a}e^{i\frac{\pi}{4}\sigma_k^z}\right)\\
        \notag=&e^{i\frac{\pi}{4}\sigma_a^x}P_a^{z,\pm}e^{-i\frac{\pi}{4}\sigma_a^z}\\
        \notag=&\sqrt{i\sigma_a^x}\frac{1}{2}(I\pm\sigma_a^z)\sqrt{-i\sigma_a^x}\\
        \notag=&\frac{1}{2}(I\pm\sqrt{i\sigma_a^x}\sigma_a^z\sqrt{-i\sigma_a^x})\\
        \notag=&\frac{1}{2}(I\pm\sigma_a^y)=P_a^{y,\pm},
    \end{align}
    从而
    \begin{align}
        P_a^{y,\pm}\lvert G\rangle=(U_a^{\tau}(G))^{\dagger}P_a^{z,\pm}U_a^{\tau}(G)\lvert G\rangle=(U_a^{\tau}(G))^{\dagger}P_a^{z,\pm}\lvert\tau_a(G)\rangle.
    \end{align}
    利用前面已证的 $P_a^{z,\pm}\lvert G\rangle=\frac{1}{\sqrt{2}}\lvert z,\pm\rangle_a\otimes U_a^{z,\pm}\lvert G-a\rangle$, 可得
    \begin{itemize}
        \item[(1)] 当 $P_a^z$ 的测量结果为 $+1$ 时,
        \begin{align}
            \notag(U_a^{\tau}(G))^{\dagger}P_a^{z,+}\lvert\tau_a(G)\rangle=&(U_a^{\tau}(G))^{\dagger}\frac{1}{\sqrt{2}}\lvert z,+\rangle_a\otimes\lvert\tau_a(G)-a\rangle\\
            \notag=&\frac{1}{\sqrt{2}}e^{i\frac{\pi}{4}\sigma_a^x}\lvert z,+\rangle_a\otimes\prod_{k\in N_a}e^{-i\frac{\pi}{4}\sigma_k^z}\lvert\tau_a(G)-a\rangle\\
            \notag=&\frac{1}{\sqrt{2}}\frac{1}{\sqrt{2}}(\lvert z,+\rangle_a+i\lvert z,-\rangle_a)\otimes\prod_{k\in N_a}e^{-i\frac{\pi}{4}\sigma_k^z}\lvert\tau_a(G)-a\rangle\\
            \notag=&\frac{1}{\sqrt{2}}\lvert y,+\rangle_a\otimes\prod_{k\in N_a}e^{-i\frac{\pi}{4}\sigma_k^z}\lvert\tau_a(G)-a\rangle\\
            =&\frac{1}{\sqrt{2}}\lvert y,+\rangle_a\otimes\prod_{k\in N_a}\sqrt{-i\sigma_k^z}\lvert\tau_a(G)-a\rangle
        \end{align}
        \item[(2)] 当 $P_a^z$ 的测量结果为 $-1$ 时,
        \begin{align}
            \notag(U_a^{\tau}(G))^{\dagger}P_a^{z,-}\lvert\tau_a(G)\rangle=&(U_a^{\tau}(G))^{\dagger}\frac{1}{\sqrt{2}}\lvert z,-\rangle_a\otimes\prod_{b\in N_a}\sigma_b^z\lvert\tau_a(G)-a\rangle\\
            \notag=&\frac{1}{\sqrt{2}}e^{i\frac{\pi}{4}\sigma_a^x}\lvert z,-\rangle_a\otimes\prod_{k\in N_a}e^{-i\frac{\pi}{4}\sigma_k^z}\prod_{b\in N_a}\sigma_b^z\lvert\tau_a(G)-a\rangle\\
            \notag=&\frac{1}{\sqrt{2}}\lvert y,-\rangle_a\otimes\prod_{k\in N_a}\sqrt{i\sigma_k^z}\lvert\tau_a(G)-a\rangle.
        \end{align}
        (忽略整体相位 $e^{i\pi/2}$)
    \end{itemize}
    综上, 得
    \begin{align}
        P_a^{y,\pm}\lvert G\rangle=\frac{1}{\sqrt{2}}\lvert y,\pm\rangle\otimes U_a^{y,\pm}\lvert\tau_a(G)-a\rangle,
    \end{align}
    其中 $U_a^{y,\pm}=\prod_{k\in N_a}\sqrt{\mp i\sigma_k^z}$.

    下证 $P_a^{x,\pm}\lvert G\rangle=\frac{1}{\sqrt{2}}\lvert x\rangle_a\otimes U_a^{x,\pm}\lvert\tau_{b_0}(\tau_a\tau_{b_0}(G)-a)\rangle$: 利用
    \begin{align}
        \notag(U_{b_0}^{\tau}(G))^{\dagger}P_a^{y,\pm}U_{b_0}^{\tau}(G)=&\left(e^{-i\frac{\pi}{4}\sigma_{b_0}^x}\prod_{c\in N_{b_0}}e^{i\frac{\pi}{4}\sigma_c^z}\right)^{\dagger}\frac{1}{2}(I\pm\sigma_a^y)\left(e^{-i\frac{\pi}{4}\sigma_{b_0}^x}\prod_{c\in N_{b_0}}e^{i\frac{\pi}{4}\sigma_c^z}\right)\\
        \notag=&\left(e^{i\frac{\pi}{4}\sigma_{b_0}^x}\prod_{c\in N_{b_0}}e^{-i\frac{\pi}{4}\sigma_c^z}\right)\frac{1}{2}(I\pm\sigma_a^y)\left(e^{-i\frac{\pi}{4}\sigma_{b_0}^x}\prod_{c\in N_{b_0}}e^{i\frac{\pi}{4}\sigma_c^z}\right)\\
        \notag=&e^{-i\frac{\pi}{4}\sigma_a^z}\frac{1}{2}(I\pm\sigma_a^y)e^{i\frac{\pi}{4}\sigma_a^z}\\
        \notag=&\sqrt{-i\sigma_a^z}\frac{1}{2}(I\pm\sigma_a^y)\sqrt{i\sigma_a^z}\\
        \notag=&\frac{1}{2}(I\pm\sqrt{-i\sigma_a^z}\sigma_a^y\sqrt{i\sigma_a^z})\\
        =&\frac{1}{2}(I\pm\sigma_a^x)=P_a^{z,\pm}\quad b_0\in N_a,
    \end{align}
    从而
    \begin{align}
        \notag P_a^{x,\pm}\lvert G\rangle=&(U_{b_0}^{\tau}(G))^{\dagger}P_a^{y,\pm}U_{b_0}^{\tau}(G)\lvert G\rangle=(U_{b_0}^{\tau}(G))^{\dagger}P_a^{y,\pm}\lvert\tau_{b_0}(G)\rangle\\
        \notag=&(U_{b_0}^{\tau}(G))^{\dagger}(U_a^{\tau}(G))^{\dagger}P_a^{z,\pm}U_a^{\tau}(G)\lvert\tau_{b_0}(G)\rangle\\
        \notag=&(U_{b_0}^{\tau}(G))^{\dagger}(U_a^{\tau}(G))^{\dagger}P_a^{z,\pm}\lvert\tau_a\tau_{b_0}(G)\rangle\\
        \notag=&(U_{b_0}^{\tau}(G))^{\dagger}(U_a^{\tau}(G))^{\dagger}\frac{1}{\sqrt{2}}\lvert z,\pm\rangle_a\otimes U_a^{z,\pm}\lvert\tau_a\tau_{b_0}(G)-a\rangle
    \end{align}
    \begin{itemize}
        \item[(1)] 当 $P_a^z$ 的测量结果为 $+1$ 时,
        \begin{align}
            \notag&(U_{b_0}^{\tau}(G))^{\dagger}(U_a^{\tau}(G))^{\dagger}\frac{1}{\sqrt{2}}\lvert z,+\rangle_a\otimes\lvert\tau_a\tau_{b_0}(G)-a\rangle\\
            \notag=&\left(e^{i\frac{\pi}{4}\sigma_{b_0}^x}\prod_{k\in N_{b_0}}e^{-i\frac{\pi}{4}\sigma_k^z}\right)\left(e^{i\frac{\pi}{4}\sigma_a^x}\prod_{j\in N_a}e^{-i\frac{\pi}{4}\sigma_j^z}\right)\frac{1}{\sqrt{2}}\lvert z,+\rangle_a\otimes\lvert\tau_a\tau_{b_0}(G)-a\rangle\\
            \notag=&\frac{1}{\sqrt{2}}e^{-i\frac{\pi}{4}\sigma_a^z}e^{i\frac{\pi}{4}\sigma_a^x}\lvert z,+\rangle_a\otimes e^{i\frac{\pi}{4}\sigma_{b_0}^x}\prod_{k\in N_{b_0}-a}e^{-i\frac{\pi}{4}\sigma_k^z}\prod_{j\in N_a}e^{-i\frac{\pi}{4}\sigma_j^z}\lvert\tau_a\tau_{b_0}(G)-a\rangle\\
            % TODO
            \notag=&\frac{1}{\sqrt{2}}e^{i\frac{\pi}{4}\sigma_a^y}\lvert z,+\rangle_a\otimes e^{i\frac{\pi}{4}\sigma_{b_0}^y}\prod_{k\in N_a-(N_{b_0}+b_0)}\sigma_k^ze^{-i\frac{\pi}{4}\sigma_{b_0}^x}\prod_{j\in N_{b_0}}e^{i\frac{\pi}{4}\sigma_j^z}\lvert\tau_a\tau_{b_0}(G)-a\rangle\\
            \notag=&\frac{1}{\sqrt{2}}\lvert x,+\rangle_a\otimes\sqrt{i\sigma_{b_0}^y}\prod_{k\in N_a-(N_{b_0}+b_0)}\sigma_k^zU_{b_0}^{\tau}\lvert\tau_a\tau_{b_0}(G)-a\rangle\\
            =&\frac{1}{\sqrt{2}}\lvert x,+\rangle_a\otimes\sqrt{i\sigma_{b_0}^y}\prod_{k\in N_a-(N_{b_0}+b_0)}\sigma_k^z\lvert\tau_{b_0}(\tau_a\tau_{b_0}(G)-a)\rangle.
        \end{align}
        \item[(2)] 当 $P_a^z$ 的测量结果为 $-1$ 时,
        \begin{align}
            \notag&(U_{b_0}^{\tau}(G))^{\dagger}(U_a^{\tau}(G))^{\dagger}\frac{1}{\sqrt{2}}\lvert z,-\rangle_a\otimes\prod_{k\in N_a}\sigma_k^z\lvert\tau_a\tau_{b_0}(G)-a\rangle\\
            \notag=&\left(e^{i\frac{\pi}{4}\sigma_{b_0}^x}\prod_{k\in N_{b_0}}e^{-i\frac{\pi}{4}\sigma_k^z}\right)\left(e^{i\frac{\pi}{4}\sigma_a^x}\prod_{j\in N_a}e^{-i\frac{\pi}{4}\sigma_j^z}\right)\frac{1}{\sqrt{2}}\lvert z,-\rangle_a\otimes\prod_{k\in N_a}\sigma_k^z\lvert\tau_a\tau_{b_0}(G)-a\rangle\\
            % TODO
            \notag=&\frac{1}{\sqrt{2}}e^{i\frac{\pi}{4}\sigma_a^y}\lvert z,-\rangle_a\otimes e^{-i\frac{\pi}{4}\sigma_{b_0}^y}\prod_{k\in N_a-(N_{b_0}+b_0)}\sigma_k^ze^{-i\frac{\pi}{4}\sigma_{b_0}^x}\prod_{j\in N_{b_0}}e^{i\frac{\pi}{4}\sigma_j^z}\lvert\tau_a\tau_{b_0}(G)-a\rangle\\
            \notag=&\frac{1}{\sqrt{2}}\lvert x,-\rangle_a\otimes e^{-i\frac{\pi}{4}\sigma_{b_0}^y}\prod_{k\in N_a-(N_{b_0}+b_0)}\sigma_k^zU_{b_0}^{\tau}\lvert\tau_a\tau_{b_0}(G)-a\rangle\\
            =&\frac{1}{\sqrt{2}}\lvert x,-\rangle\otimes\sqrt{-i\sigma_{b_0}^y}\prod_{k\in N_a-(N_{b_0}+b_0)}\sigma_k^z\lvert\tau_{b_0}(\tau_a\tau_{b_0}(G)-a)\rangle.
        \end{align}
    \end{itemize}
    综上, 得
    \begin{align}
        P_a^{x,\pm}\lvert G\rangle=\frac{1}{\sqrt{2}}\lvert x,\pm\rangle\otimes U_a^{x,\pm}\lvert\tau_{b_0}(\tau_a\tau_{b_0}(G)-a)\rangle,
    \end{align}
    其中 $U_a^{x,\pm}=\sqrt{\pm i\sigma_{b_0}^y}\prod_{k\in N_a-(N_{b_0}+b_0)}\sigma_k^z$.
\end{pf}

\begin{prob}
    证明第 18 页中的对易过程.
\end{prob}
\begin{sol}
    \begin{align}
        He^{i\alpha_4Z/2}Z^{s_4}
    \end{align}
\end{sol}

\begin{prob}
    推导 22-23 页中 CNOT 门构造中的关系式.
\end{prob}
\begin{sol}
    
\end{sol}
\end{document}