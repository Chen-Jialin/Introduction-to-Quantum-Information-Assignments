\documentclass{assignment}
\ProjectInfos{量子信息导论}{PHYS5251P}{2021-2022 学年第一学期}{第四次作业}{截止日期:2021. 11. 24(周三)}{陈稼霖}[https://github.com/Chen-Jialin]{SA21038052}

\begin{document}
\begin{prob}
    Alice 和 Bob 选择 B92 方案来建立量子秘钥序列. Alice 选择两种态: $\lvert\psi_1\rangle=\lvert 0\rangle$, $\lvert\psi_2\rangle=\frac{1}{\sqrt{2}}(\lvert 0\rangle+\lvert 1\rangle)$, 分别以 $1/2$ 的概率发送给 Bob, Bob 分别以 $1/2$ 的几率选择基 $\{\lvert 0\rangle,\lvert 1\rangle\}$ 和 基 $\{1/\sqrt{2}(\lvert 0\rangle+\lvert 1\rangle),1/\sqrt{2}(\lvert 0\rangle-\lvert 1\rangle)\}$ 对收到的态进行正交测量.
    \begin{itemize}
        \item[(1)] 请论述 Alice 和 Bob 将遵从怎样的经典通信协议来建立秘钥;
        \item[(2)] 假定存在一个窃听者, 该窃听者试图以概率克隆的方式对该秘钥建立过程进行攻击. 则以下的几组克隆概率中, 哪几组在理论上是可行的 (括号中第一个数表示成功地克隆出 $\lvert\psi_1\rangle$ 的概率, 第二个数表示成功地克隆出 $\lvert\psi_2\rangle$ 的概率). 并给出证明.
        \[
            \left(\frac{2-\sqrt{2}}{2},\frac{2-\sqrt{2}}{2}\right),\quad(1,0.1),\quad(0.5,0.5),\quad(0.7,0.7),\quad(0.9,0.9)
        \]
        \item[(3)] 窃听者如果克隆失败, 他会随机发送 $\lvert\psi_1\rangle$ 或 $\lvert\psi_2\rangle$ 给 Bob (分别以 $1/2$ 的概率). 如果窃听者选择以上几组中最优的克隆方案进行攻击, 则作为 Alice 和 Bob, 他们至少要公开对照多少组数据, 均检验无误, 才能确保该秘钥的安全性达到 $99\%$ 以上?
    \end{itemize}
\end{prob}
\begin{sol}
    \begin{itemize}
        \item[(1)] Bob 保留测得为 $\lvert 1\rangle$ 或 $\frac{1}{\sqrt{2}}(\lvert 0\rangle-\lvert 1\rangle)$ 的结果, 而抛弃其他结果, 并将保留的结果在序列中的位置告诉 Alice, 从而建立秘钥. 具体来说, 分为以下 6 种情况:

        \begin{table}[h]
            \centering
            \resizebox{\textwidth}{!}{%
            \begin{tabular}{|c|ccc|ccc|}
            \hline
            Alice 发送的量子态 & \multicolumn{3}{c|}{$\lvert\psi_1\rangle=\lvert 0\rangle$} & \multicolumn{3}{c|}{$\lvert\psi_2\rangle=\frac{1}{\sqrt{2}}(\lvert 0\rangle+\lvert 1\rangle)$} \\ \hline
            Bob 选择的测量基 & \multicolumn{1}{c|}{$\{\lvert 0\rangle,\lvert 1\rangle\}$} & \multicolumn{2}{c|}{$\{\frac{1}{\sqrt{2}}(\lvert 0\rangle+\lvert 1\rangle),\frac{1}{\sqrt{2}}(\lvert 0\rangle-\lvert 1\rangle)\}$} & \multicolumn{2}{c|}{$\{\lvert 0\rangle,\lvert 1\rangle\}$} & $\{\frac{1}{\sqrt{2}}(\lvert 0\rangle+\lvert 1\rangle),\frac{1}{\sqrt{2}}(\lvert 0\rangle-\lvert 1\rangle)\}$ \\ \hline
            Bob 的测量结果 & \multicolumn{1}{c|}{$\lvert 0\rangle$} & \multicolumn{1}{c|}{$\frac{1}{\sqrt{2}}(\lvert 0\rangle+\lvert 1\rangle)$} & $\frac{1}{2}(\lvert 0\rangle-\lvert 1\rangle)$ & \multicolumn{1}{c|}{$\lvert 0\rangle$} & \multicolumn{1}{c|}{$\lvert 1\rangle$} & $\frac{1}{\sqrt{2}}(\lvert 0\rangle-\lvert 1\rangle)$ \\ \hline
            是否保留 & \multicolumn{1}{c|}{否} & \multicolumn{1}{c|}{否} & 是 & \multicolumn{1}{c|}{否} & \multicolumn{1}{c|}{是} & 否 \\ \hline
            生成的秘钥 & \multicolumn{1}{c|}{} & \multicolumn{1}{c|}{} & 0 & \multicolumn{1}{c|}{} & \multicolumn{1}{c|}{1} &  \\ \hline
            \end{tabular}%
        }
        \end{table}

        建立秘钥后, Alice 和 Bob 选择部分秘钥进行比较, 以检查是否有窃听.
        \item[(2)] 定义
        \begin{align}
            X^{(1)}=&\begin{pmatrix}
                \langle\psi_1\vert\psi_1\rangle&\langle\psi_1\vert\psi_2\rangle\\
                \langle\psi_2\vert\psi_1\rangle&\langle\psi_2\vert\psi_2\rangle
            \end{pmatrix}=\begin{pmatrix}
                1&\frac{1}{\sqrt{2}}\\
                \frac{1}{\sqrt{2}}&1
            \end{pmatrix},\\
            X^{(2)}=&\begin{pmatrix}
                (\langle\psi_1\vert\psi_1\rangle)^2&(\langle\psi_1\vert\psi_2\rangle)^2\\
                (\langle\psi_2\vert\psi_1\rangle)^2&(\langle\psi_2\vert\psi_2\rangle)^2
            \end{pmatrix}=\begin{pmatrix}
                1&\frac{1}{2}\\
                \frac{1}{2}&1
            \end{pmatrix}.
        \end{align}
        对克隆概率 $(r_1,r_2)$,
        \begin{align}
            \Gamma=&\begin{pmatrix}
                r_1&0\\
                0&r_2
            \end{pmatrix},\\
            X^{(1)}-\sqrt{\Gamma}X^{(2)}\sqrt{\Gamma}=&\begin{pmatrix}
                1-r_1&\frac{\sqrt{2}-\sqrt{r_1r_2}}{2}\\
                \frac{\sqrt{2}-\sqrt{r_1r_2}}{2}&1-r_2
            \end{pmatrix}.
        \end{align}
        若 $X^{(1)}-\sqrt{\Gamma}X^{(2)}\sqrt{\Gamma}$ 的顺序主子式的行列式均为正, 即
        \begin{align}
            1-r_1>0,\quad\abs{X^{(1)}-\sqrt{\Gamma}X^{(2)}\sqrt{\Gamma}}=\begin{vmatrix}
                1-r_1&\frac{\sqrt{2}-\sqrt{r_1r_2}}{2}\\
                \frac{\sqrt{2}-\sqrt{r_1r_2}}{2}&1-r_2
            \end{vmatrix}=(1-r_1)(1-r_2)-\left(\frac{\sqrt{2}-\sqrt{r_1r_2}}{2}\right)^2>0,
        \end{align}
        则 $X^{(1)}-\sqrt{\Gamma}X^{(2)}\sqrt{\Gamma}$ 正定, 克隆概率 $(r_1,r_2)$ 可行, 否则 $X^{(1)}-\sqrt{\Gamma}X^{(2)}\sqrt{\Gamma}$ 非正定, 克隆概率 $(r_1,r_2)$ 不可行.
        \begin{itemize}
            \item 对克隆概率 $\left(\frac{2-\sqrt{2}}{2},\frac{2-\sqrt{2}}{2}\right)$,
            \[
                1-r_1=\frac{\sqrt{2}}{2}>0,\quad\abs{X^{(1)}-\sqrt{\Gamma}X^{(2)}\sqrt{\Gamma}}=(1-r_1)(1-r_2)-\left(\frac{\sqrt{2}-\sqrt{r_1r_2}}{2}\right)^2=\frac{6\sqrt{2}-7}{8}>0,
            \]
            故 $X^{(1)}-\sqrt{\Gamma}X^{(2)}\sqrt{\Gamma}$ 正定, 克隆概率 $\left(\frac{2-\sqrt{2}}{2},\frac{2-\sqrt{2}}{2}\right)$ 可行.
            \item 对克隆概率 $(1,0.1)$,
            \[
                1-r_1=0,\quad\abs{X^{(1)}-\sqrt{\Gamma}X^{(2)}\sqrt{\Gamma}}=(1-r_1)(1-r_2)-\left(\frac{\sqrt{2}-\sqrt{r_1r_2}}{2}\right)^2=-\left(\frac{\sqrt{2}-\sqrt{0.1}}{2}\right)^2<0
            \]
            故 $X^{(1)}-\sqrt{\Gamma}X^{(2)}\sqrt{\Gamma}$ 非正定, 克隆概率 $\left(\frac{2-\sqrt{2}}{2},\frac{2-\sqrt{2}}{2}\right)$ 不可行.
            \item 对克隆概率 $(0.5,0.5)$,
            \[
                1-r_1=0.5>0,\quad\abs{X^{(1)}-\sqrt{\Gamma}X^{(2)}\sqrt{\Gamma}}=(1-r_1)(1-r_2)-\left(\frac{\sqrt{2}-\sqrt{r_1r_2}}{2}\right)^2=\frac{4\sqrt{2}-5}{16}>0,
            \]
            故 $X^{(1)}-\sqrt{\Gamma}X^{(2)}\sqrt{\Gamma}$ 正定, 克隆概率 $(0.5,0.5)$ 可行.
            \item 对克隆概率 $(0.7,0.7)$,
            \[
                1-r_1=0.3>0,\quad\abs{X^{(1)}-\sqrt{\Gamma}X^{(2)}\sqrt{\Gamma}}=(1-r_1)(1-r_2)-\left(\frac{\sqrt{2}-\sqrt{r_1r_2}}{2}\right)^2=0.3^2-\left(\frac{\sqrt{2}-0.7}{2}\right)^2<0,
            \]
            故 $X^{(1)}-\sqrt{\Gamma}X^{(2)}\sqrt{\Gamma}$ 非正定, 克隆概率 $(0.7,0.7)$ 不可行.
            \item 对克隆概率 $(0.9,0.9)$,
            \[
                1-r_1=0.1>0,\quad\abs{X^{(1)}-\sqrt{\Gamma}X^{(2)}\sqrt{\Gamma}}=(1-r_1)(1-r_2)-\left(\frac{\sqrt{2}-\sqrt{r_1r_2}}{2}\right)^2=0.1^2-\left(\frac{\sqrt{2}-0.9}{2}\right)^2<0,
            \]
            故 $X^{(1)}-\sqrt{\Gamma}X^{(2)}\sqrt{\Gamma}$ 非正定, 克隆概率 $(0.9,0.9)$ 不可行.
        \end{itemize}
        \item[(3)] 以上几组方案中, 最优的克隆概率为 $(0.5,0.5)$, 即无论 Alice 发送 $\lvert\psi_1\rangle$ 和 $\lvert\psi_2\rangle$ 中的任何一种, 窃听者 Eve 均有 $0.5$ 的概率克隆成功, 有 $1-0.5=0.5$ 的概率克隆失败. 若 Eve 克隆成功, 则该次窃听不会被发现; 若 Eve 克隆失败, 则 Eve 随机发送的 $\lvert\psi_1\rangle$ 和 $\lvert\psi_2\rangle$ 中的一种, 有 $\frac{1}{2}$ 的概率和 Alice 发送的态相同, 有 $\frac{1}{2}$ 的概率和 Alice 发送的态不同. 若 Eve 随机选择的态和 Alice 发送的态相同, 则窃听仍不会被发现; 若 Eve 随机发送的态和 Alice 发送的态不同, 则 Bob 收到后, 若能成功生成秘钥, 则该秘钥必然错误, 从而在与 Alice 的比对中发现窃听. 综上, 每位秘钥的对比都有 $\frac{3}{4}$ 的概率发现 Eve 的窃听, 要使秘钥的安全性达到 $99\%$ 以上, 即
        \begin{align}
            P_d=1-\left(\frac{3}{4}\right)^n>0.99
        \end{align}
        则至少 $n\geq 17$, 即至少需要公开对照 $17$ 组数据, 均检验无误, 才能确保该秘钥的安全性达到 $99\%$ 以上.
    \end{itemize}
\end{sol}

\begin{prob}
    给出高维空间量子 teleportation 的数学证明.
\end{prob}
\begin{pf}
    假设 Alice 处有一带传送的粒子, 标号为 1, 处于 $N$ 维未知量子态
    \begin{align}
        \lvert\chi\rangle=\sum_{k=0}^{N-1}c_k\lvert k\rangle,
    \end{align}
    Alice 和 Bob 共享一对处于 $N$ 维最大纠缠态
    \begin{align}
        \lvert\psi_{00}\rangle=\frac{1}{\sqrt{N}}\sum_{j=0}^{N-1}\lvert j\rangle\lvert j\rangle.
    \end{align}
    的粒子 2 和 3, 其中粒子 2 位于 Alice 处, 粒子 3 处于 Bob 处.
    三个粒子的总量子态为
    \begin{align}
        \lvert\chi\rangle\lvert\psi_{00}\rangle=\frac{1}{\sqrt{N}}\sum_{k,j=0}^{N-1}c_k\lvert k\rangle\lvert j\rangle\lvert j\rangle.
    \end{align}
    取一组两粒子的纠缠态正交基
    \begin{align}
        \lvert\psi_{mn}\rangle=\frac{1}{\sqrt{N}}\sum_{l=0}^{N-1}e^{i2\pi ln/N}\lvert l\rangle\lvert(l+m)\mod N\rangle,
    \end{align}
    和幺正变换
    \begin{align}
        U_{mn}=\sum_{l=0}^{N-1}e^{i2\pi ln/N}\lvert j\rangle\langle(l+m)\mod N\rvert,
    \end{align}
    有
    \begin{align}
        \notag U_{mn}^{\dagger}\lvert\chi\rangle=&\sum_{l,k=0}^{N-1}c_ke^{-i2\pi ln/N}\lvert(l+m)\mod N\rangle\langle l\vert k\rangle,\\
        \notag=&\sum_{l,k=0}^{N-1}c_ke^{-i2\pi ln/N}\lvert(l+m)\mod N\rangle\delta_{lk}\\
        =&\sum_{k=0}^{N-1}c_ke^{-i2\pi kn/N}\lvert(k+m)\mod N\rangle,\\
        \lvert\psi_{mn}\rangle\otimes U_{mn}^{\dagger}\lvert\chi\rangle=&\frac{1}{\sqrt{N}}\sum_{l,k=0}^{N-1}c_ke^{i2\pi(l-k)n/N}\lvert l\rangle\lvert(l+m)\mod N\rangle\lvert(k+m)\mod N\rangle,\\
        \notag\frac{1}{\sqrt{N}}\sum_{m,n=0}^{N-1}\lvert\psi_{mn}\rangle\otimes U_{mn}^{\dagger}\lvert\chi\rangle=&\frac{1}{\sqrt{N}}\sum_{m,n,l,k=0}^{N-1}c_ke^{i2\pi(l-k)n/N}\lvert l\rangle\lvert(l+m)\mod N\rangle\lvert(k+m)\mod N\rangle\\
        \notag&(\because\frac{1}{N}\sum_{n=0}^{N-1}e^{-i2\pi(k-l)n/N}=\delta_{lk})\\
        \notag=&\sum_{m,l,k=0}^{N-1}c_k\delta_{lk}\lvert l\rangle\lvert(l+m)\mod N\rangle\lvert(k+m)\mod N\rangle\\
        \notag=&\sum_{m,k=0}^{N-1}c_k\lvert k\rangle\lvert(k+m)\mod N\rangle\lvert(k+m)\mod N\rangle\\
        \notag&(\text{令 }j=(k+m)\mod N)\\
        \notag=&\sum_{k=0}^{N-1}c_k\lvert k\rangle\otimes\sum_{j=0}^{N-1}\lvert j\rangle\lvert j\rangle\\
        =&\lvert\psi_{00}\rangle\otimes\lvert\chi\rangle.
    \end{align}
    因此只需要 Alice 对粒子 1 和 2 以 $\{\lvert\psi_{mn}\rangle\}$ 为基做正交测量, 并将测量结果以经典通讯方式告知 Bob, 然后 Bob 对 粒子 3 做相应的幺正操作 $U_{mn}$, 就可以在粒子 3 上复现为原来待传粒子 1 的状态, 此即高维的 teleportation.
\end{pf}

\begin{prob}
    混合纠缠态 $\rho(\lambda)=(1-\lambda)\lvert\psi^-\rangle\langle\psi^-\rvert+\frac{\lambda}{4}I\otimes I$
    \begin{itemize}
        \item[a)] 求标准 teleportation 的保真度, 并且, 当 $\lambda$ 达到多少时, 保真度将优于经典极限? (所谓经典极限是指: A 方随机选择一组测量基进行测量, 并将测量结果通过经典信道通知 B, B 根据 A 的测量结果进行态制备.)
        \item[b)] 计算 $\text{Prob}(\uparrow(\vec{n})\uparrow(\vec{m}))=\tr(E_A(\vec{n})E_A(\vec{m})\rho(\lambda))$\\
        $E(\vec{n})$ 是 Alice 的比例投影到 $\lvert\uparrow(\vec{n})\rangle$ 上的投影子.
    \end{itemize}
\end{prob}
\begin{sol}
    \begin{itemize}
        \item[a)] 混合纠缠态 $\rho(\lambda)$ 可视为有 $1-\lambda$ 的概率处于 $\lvert\psi^-\rangle\langle\psi^-\rvert$ 的纠缠态, 而有 $\lambda$ 的概率处于可分混合态 $\frac{1}{4}I\times I$ (无纠缠). $\lvert\psi^-\rangle\langle\psi^-\rvert$ 可用于准确地传递待传态, 保真度为 $1$, 而 $\frac{1}{4}I\otimes I$ 无法用于传态, 相当于传了一个随机的量子态, 仅有 $\frac{1}{2}$ 的保真度, 故利用 $\rho(\lambda)$ 进行 teleportation, 保真度为
        \begin{align}
            F=(1-\lambda)+\frac{\lambda}{2}=1-\frac{\lambda}{2}.
        \end{align}

        若 A 随机选择一组测量基进行测量, 并将测量结果通过经典信道通知 B, B 根据 A 的测量结果进行态制备, 则保真度 (经典极限) 为 $\frac{2}{3}$.

        \[
            F=1-\frac{\lambda}{2}>\frac{2}{3}\Longrightarrow\lambda<\frac{2}{3},
        \]
        故当 $\lambda<\frac{2}{3}$, 则保真度优于经典极限.
        \item[b)] 设 $\vec{n}=(\sin\theta\cos\phi,\sin\theta\sin\phi,\sin\theta)$, 则
        \begin{align}
            \notag E_A(\vec{n})=&\left(\cos\frac{\theta}{2}\lvert 0\rangle+e^{i\phi}\sin\frac{\theta_1}{2}\lvert 1\rangle\right)\left(\cos\frac{\theta}{2}\langle 0\rvert+e^{-i\phi}\sin\frac{\theta}{2}\langle 1\rvert\right)=\begin{pmatrix}
                \cos^2\frac{\theta}{2}&e^{i\phi}\cos\frac{\theta}{2}\sin\frac{\theta}{2}\\
                e^{-i\phi}\cos\frac{\theta}{2}\sin\frac{\theta}{2}&\sin^2\frac{\theta}{2}
            \end{pmatrix}\\
            =&\frac{1}{2}\begin{pmatrix}
                1+\cos\theta&e^{i\phi}\sin\theta\\
                e^{-i\phi}\sin\theta&1-\cos\theta
            \end{pmatrix}=\frac{1}{2}(I+\vec{n}\cdot\vec{\sigma}),
        \end{align}
        其中 $\vec{\sigma}=\begin{pmatrix}
            \sigma_1&\sigma_2&\sigma_3
        \end{pmatrix}^T$.
        同理,
        \begin{align}
            E_A(\vec{m})=\frac{1}{2}(I+\vec{m}\cdot\vec{\sigma}).
        \end{align}
        \begin{align}
            \notag\text{Prob}(\uparrow(\vec{n})\uparrow(\vec{m}))=&\tr(E_A(\vec{n})E_A(\vec{m})\rho(\lambda))=\tr\left(\frac{1}{2}(I+\vec{n}\cdot\vec{\sigma})\otimes\frac{1}{2}(I+\vec{m}\cdot\vec{\sigma})((1-\lambda)\lvert\psi^-\rangle\langle\psi^-\rvert+\frac{\lambda}{4}I\otimes I)\right)\\
            \notag=&\frac{1-\lambda}{4}\tr((I+\vec{n}\cdot\vec{\sigma})\otimes(I+\vec{m}\cdot\vec{\sigma})\lvert\psi^-\rangle\langle\psi^-\rvert)+\frac{\lambda}{16}\tr((I+\vec{n}\cdot\vec{\sigma})\otimes(I+\vec{m}\cdot\vec{\sigma})(I\otimes I))\\
            \notag=&\frac{1-\lambda}{4}\tr((I+\vec{n}\cdot\vec{\sigma})\otimes(I+\vec{m}\cdot\vec{\sigma})\lvert\psi^-\rangle\langle\psi^-\rvert)+\frac{\lambda}{16}\tr_A(I+\vec{n}\cdot\vec{\sigma})\tr_B(I+\vec{m}\cdot\vec{\sigma})\\
            \notag&(\because\tr\left(\frac{1}{2}(I+\vec{n}\cdot\vec{\sigma})=1\right))\\
            \notag=&\frac{1-\lambda}{4}\langle\psi^-\rvert(I+\vec{n}\cdot\vec{\sigma})\otimes(I+\vec{m}\cdot\vec{\sigma})\lvert\psi^-\rangle+\frac{\lambda}{4}\\
            \notag=&\frac{1-\lambda}{4}\langle\psi^-\rvert I\otimes I\lvert\psi^-\rangle+\frac{1-\lambda}{4}\langle\psi^-\rvert\vec{n}\cdot\vec{\sigma}\otimes I\lvert\psi^-\rangle+\frac{1-\lambda}{4}\langle\psi^-\rvert I\otimes\vec{m}\cdot\vec{\sigma}\lvert\psi^-\rangle\\
            \notag&+\frac{1-\lambda}{4}\langle\psi^-\rvert\vec{n}\cdot\vec{\sigma}\otimes\vec{m}\cdot\vec{\sigma}\lvert\psi^-\rangle+\frac{\lambda}{4}\\
            =&\frac{1-\lambda}{4}\langle\psi^-\rvert\vec{n}\cdot\vec{\sigma}\otimes I\lvert\psi^-\rangle+\frac{1-\lambda}{4}\langle\psi^-\rvert I\otimes\vec{m}\cdot\vec{\sigma}\lvert\psi^-\rangle+\frac{1-\lambda}{4}\langle\psi^-\rvert\vec{n}\cdot\vec{\sigma}\otimes\vec{m}\cdot\vec{\sigma}\lvert\psi^-\rangle+\frac{1}{4}
        \end{align}
        其中
        \begin{align}
            \notag\langle\psi^-\rvert\vec{n}\cdot\vec{\sigma}\otimes I\lvert\psi^-\rangle=&\frac{1}{\sqrt{2}}(\langle 01\rvert-\langle 10\rvert)(\vec{n}\cdot\vec{\sigma}\otimes I)\frac{1}{\sqrt{2}}(\lvert 01\rangle-\lvert 10\rangle)\\
            \notag=&\frac{1}{2}(\langle 01\rvert-\langle 10\rvert)((n_1\sigma_1+n_2\sigma_2+n_3\sigma_3)\otimes I)(\lvert 01\rangle-\lvert 10\rangle)\\
            \notag=&\frac{1}{2}(\langle 01\rvert-\langle 10\rvert)[n_1(\lvert 11\rangle-\lvert 00\rangle)+n_2(-i\lvert 11\rangle-i\lvert 00\rangle)+n_3(\lvert 01\rangle+\lvert 10\rangle)]\\
            =&0,
        \end{align}
        同理,
        \begin{align}
            \langle\psi^-\rvert I\otimes\vec{m}\cdot\vec{\sigma}\otimes I\lvert\psi^-\rangle=0,
        \end{align}
        此外,
        \begin{align}
            \notag&\langle\psi^-\rvert(\vec{n}\cdot\vec{\sigma}\otimes\vec{m}\cdot\vec{\sigma})\lvert\psi^-\rangle\\
            \notag=&\frac{1}{\sqrt{2}}(\langle 01\rvert-\langle 10\rvert)[n_1m_1\sigma_1\otimes\sigma_1+n_1m_2\sigma_1\otimes\sigma_2+n_1m_3\sigma_1\otimes\sigma_3+n_2m_1\sigma_2\otimes\sigma_1+n_2m_2\sigma_2\otimes\sigma_2+n_2m_3\sigma_2\otimes\sigma_3\\\notag&+n_3m_1\sigma_3\otimes\sigma_1+n_3m_2\sigma_3\otimes\sigma_2+n_3m_3\sigma_3\otimes\sigma_3]\frac{1}{\sqrt{2}}(\lvert 01\rangle-\lvert 10\rangle)\\
            =&n_1m_1+n_2m_2+n_3m_3=\cos\theta,
        \end{align}
        其中 $\theta$ 为 $\vec{n}$ 与 $\vec{m}$ 之间的夹角.
        因此,
        \begin{align}
            \text{Prob}(\uparrow(\vec{n})\uparrow(\vec{m}))=\frac{1-\lambda}{4}\cos\theta+\frac{1}{4}.
        \end{align}
    \end{itemize}
\end{sol}
\end{document}