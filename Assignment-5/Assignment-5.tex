\documentclass{assignment}
\ProjectInfos{量子信息导论}{PHYS5251P}{2021-2022 学年第一学期}{第五次作业(量子计算第一章)}{截止日期:2021. 12. 22(周三)}{陈稼霖}[https://github.com/Chen-Jialin]{SA21038052}

\begin{document}
\begin{prob}
    按图灵机中两态转移矩阵的定义 (PPT 7、8 页), 说明 $4+5$ 的算法过程.
\end{prob}
\begin{sol}
    加法的两态转移矩阵如下:
    \begin{table}[h]
        \centering
        \begin{tabular}{|c|cc|}
        \hline
        \multirow{2}{*}{状态} & \multicolumn{2}{c|}{扫描到的符号} \\ \cline{2-3} 
         & \multicolumn{1}{c|}{1} & 0 \\ \hline
        $S_1$ & \multicolumn{1}{c|}{($S_2$,0;R)} & ($S_1$,0;R) \\ \hline
        $S_2$ & \multicolumn{1}{c|}{($S_2$,1;R)} & (halt,1;R) \\ \hline
        Halt & \multicolumn{1}{c|}{停止} & 停止 \\ \hline
        \end{tabular}
    \end{table}

    欲计算 $4+5$, 初始时刻在记录带上分别有 $4$ 个连续的 1 和 $5$ 个连续的 1, 两组 1 用 0 分隔开, 按照上述转移矩阵, 计算过程如下:
    \begin{table}[h]
        \centering
        \begin{tabular}{|c|c|c|c|}
        \hline
        步骤序号 & 图灵机状态 & 记录带内容 (划线处代表读写头所在位置) & 跃迁方式 \\ \hline
        0 & $S_1$ & 0 0 \uline{1} 1 1 1 0 1 1 1 1 1 0 0 & ($S_1$,1)$\rightarrow$($S_2$,0;R) \\ \hline
        1 & $S_2$ & 0 0 0 \uline{1} 1 1 0 1 1 1 1 1 0 0 & ($S_2$,1)$\rightarrow$($S_2$,1;R) \\ \hline
        2 & $S_2$ & 0 0 0 1 \uline{1} 1 0 1 1 1 1 1 0 0 & ($S_2$,1)$\rightarrow$($S_2$,1;R) \\ \hline
        3 & $S_2$ & 0 0 0 1 1 \uline{1} 0 1 1 1 1 1 0 0 & ($S_2$,1)$\rightarrow$($S_2$,1;R) \\ \hline
        4 & $S_2$ & 0 0 0 1 1 1 \uline{0} 1 1 1 1 1 0 0 & ($S_2$,0)$\rightarrow$(halt,1;R) \\ \hline
        5 & Halt & 0 0 0 1 1 1 \uline{1} 1 1 1 1 1 0 0 &  \\ \hline
        \end{tabular}
    \end{table}

    最后得到记录带上有 $9$ 个连续的 1, 这说明 $4+5$ 的计算结果为 $9$.
\end{sol}

\begin{prob}
    给出第 67-68 页中的 Pauli 算符与标准 Pauli 算符之间的关系.
\end{prob}
\begin{sol}
    % Toffoli 门可视为受双比特控制的非门,
    % \begin{align}
    %     \text{Toffoli}=(\sigma_x)_{67}=\begin{pmatrix}
    %         0_{6\times 6}&0\\
    %         0&\sigma_x
    %     \end{pmatrix}.
    % \end{align}
    % 利用基对换操作, 可得
    % \begin{align}
    %     \sigma_{mn}=P(\sigma_x)
    % \end{align}
    % 其中
    % \begin{align}
    %     P=(6m)(7n),
    % \end{align}
    % 即将第 $6$ 行与第 $m$ 行交换, 第 $7$ 行与第 $n$ 行交换.

    % 利用 $\sigma_x$ 型算符的对易得 $\sigma_y$ 型算符:
    % \begin{align}
    %     [(\sigma_x)_{nm},(\sigma_x)_{mk}]=i(\sigma_y)_{mk}.
    % \end{align}
    % 而 $\sigma_y$ 型算符之间可通过基对换操作相互转化.
    % 以 $(\sigma_y)_{57}$ 的生成为例:
    % \begin{align}
    %     \notag[(\sigma_x)_{56},(\sigma_x)_{57}]=&(\sigma_x)_{56}(\sigma_x)_{67}-(\sigma_x)_{67}(\sigma_x)_{56}\\
    %     \notag
    %     =&\begin{pmatrix}
    %         0_{5\times 5}\\
    %         &0&1&0\\
    %         &1&0&0\\
    %         &0&0&0
    %     \end{pmatrix}\begin{pmatrix}
    %         0_{5\times 5}\\
    %         &0&0&0\\
    %         &0&0&1\\
    %         &0&1&0
    %     \end{pmatrix}-\begin{pmatrix}
    %         0_{5\times 5}\\
    %         &0&0&0\\
    %         &0&0&1\\
    %         &0&1&0
    %     \end{pmatrix}\begin{pmatrix}
    %         0_{5\times 5}\\
    %         &0&1&0\\
    %         &1&0&0\\
    %         &0&0&0
    %     \end{pmatrix}\\
    %     &=\begin{pmatrix}
    %         0_{5\times 5}\\
    %         &0&0&1\\
    %         &0&0&0\\
    %         &-1&0&0
    %     \end{pmatrix}=i\begin{pmatrix}
    %         0_{5\times 5}\\
    %         &0&0&-i\\
    %         &0&0&0\\
    %         &i&0&0
    %     \end{pmatrix}=i(\sigma_y)_{57}.
    % \end{align}

    % 利用 $\sigma_x$ 型算符和 $\sigma_y$ 型算符的对易得 $\sigma_z$ 型算符:
    % \begin{align}
    %     [(\sigma_x)_{mn},(\sigma_y)_{mn}]=i(\sigma_z)_{mn}.
    % \end{align}
    % 以 $(\sigma_z)_{67}$ 的生成为例:
    % \begin{align}
    %     \notag[(\sigma_x)_{67},(\sigma_y)_{67}]=&(\sigma_x)_{67}(\sigma_y)_{67}-(\sigma_y)_{67}(\sigma_x)_{67}\\
    %     \notag=&\begin{pmatrix}
    %         0_{6\times 6}\\
    %         &0&1\\
    %         &1&0
    %     \end{pmatrix}\begin{pmatrix}
    %         0_{6\times 6}\\
    %         &0&-i\\
    %         &i&0
    %     \end{pmatrix}-\begin{pmatrix}
    %         0_{6\times 6}\\
    %         &0&-i\\
    %         &i&0
    %     \end{pmatrix}\begin{pmatrix}
    %         0_{6\times 6}\\
    %         &0&1\\
    %         &1&0
    %     \end{pmatrix}\\
    %     =&\begin{pmatrix}
    %         0_{6\times 6}\\
    %         &2i&0\\
    %         &0&-2i
    %     \end{pmatrix}=2i(\sigma_z)_{67}.
    % \end{align}

    一方面, 在置零的辅助量子比特的辅助下, 显然可以利用 $1$ $(\sigma_k)_{mn}$ ($k=x,y,z$; $m,n=0,1,\cdots,7$) 实现标准 Pauli 算符.

    另一方面, 系统的厄米算符均可用标准 Pauli 算符的直积展开, 即 $H=\sum_{i,j,k=0}^3a_{ijk}\sigma_i^1\otimes\sigma_j^2\otimes\sigma_k^3$, 包括 $(\sigma_k)_{mn}$.\\
    例如, 对 $(\sigma_x)_{67}$,\\
    \begin{align}
        (\sigma_x)_{67}=\sum_{i,j,k=0}^3a_{ijk}\sigma_i^1\otimes\sigma_j^2\otimes\sigma_k^3,
    \end{align}
    其中
    \begin{align}
        a_{ijk}=\frac{1}{8}\tr[(\sigma_x)_{67})(\sigma_i^1\otimes\sigma_j^2\otimes\sigma_k^3)],
    \end{align}
    故
    \begin{align}
        \notag(\sigma_x)_{67}&=\frac{3}{4}I\otimes I\otimes I+\frac{1}{4}I\otimes I\otimes\sigma_x+\frac{1}{4}I\otimes\sigma_z\otimes I-\frac{1}{4}I\otimes\sigma_z\otimes\sigma_x\\
        &+\frac{1}{4}\sigma_z\otimes I\otimes I-\frac{1}{4}\sigma_z\otimes I\otimes\sigma_x-\frac{1}{4}\sigma_z\otimes\sigma_z\otimes I+\frac{1}{4}\sigma_z\otimes\sigma_z\otimes\sigma_x.
    \end{align}
    进而可以利用基变换操作得到 $(\sigma_x)_{mn}$ 型算符, 利用 $(\sigma_x)_{mn}$ 间的对易可得 $(\sigma_y)_{mn}$ 型算符:
    \begin{align}
        [(\sigma_x)_{mn},(\sigma_x)_{nk}]=-(\sigma_y)_{mk},
    \end{align}
    利用 $\sigma_x$ 型算符和 $\sigma_y$ 型算符的对易得 $\sigma_z$ 型算符:
    \begin{align}
        [(\sigma_x)_{mn},(\sigma_y)_{mn}]=2i(\sigma_z)_{mn}.
    \end{align}

    综上, $(\sigma_k)_{mn}$ 型的算符与 Pauli 可相互转化.
\end{sol}

\begin{prob}
    按参数个数估计用 Deutsch 门逼近任意 $n$ 比特幺正变换时所需的门个数.
\end{prob}
\begin{sol}
    要求用 Deutsch 门逼近任意 $n$ 比特幺正变换时所需的门个数, 等价于求对任意 $n$ 比特量子态, 利用 Deutsch 门实现 $\lvert\psi\rangle\rightarrow\lvert 1\rangle^{\otimes n}$ 所需的 Deutsch 门个数.

    设 $\lvert\psi\rangle=\sum_{i=0}^{2^n-1}c_i\lvert i\rangle$. 由于利用 Deutsch 门中的 $R$ 算符, 可以实现普适的单比特幺正变换. 我们总可以利用 $n$ 量子比特 Deutsch 门得到作用于基矢 $\lvert 2^n-2\rangle,\lvert 2^n-1\rangle$ 的量子门 $U_{2^n-2,2^n-1}$, 使得
    \begin{align}
        U_{2^n-2,2^n-1}\lvert\psi\rangle=\sum_{i=0}^{2^n-3}c_i\lvert i\rangle+\sqrt{\abs{c_{2^n-2}}^2+\abs{c_{2^n-1}}^2}\lvert 2^n-1\rangle.
    \end{align}

    在适当的基置换操作下, 我们可以构造出适当的 $U_{i,2^n-1}$ 门, 并通过 $U=\prod_{i=0}^{2^n-1}U_{i,2^n-1}$ 的组合构造出所需的幺正变换 $U$, 使得 $U\lvert\psi\rangle=\lvert 1\rangle^{\otimes 1}$.

    在上述过程中需要用到 $2^n$ 个 $n$ 量子比特 Deutsch 门和 $2^n$ 个基置换操作, 而实现每个 $n$ 量子比特 Deutsch 门需要 $O(n)$ 个 Deutsch 门级级联以及 $O(n)$ 个置零的辅助量子比特. 综上, 用 Deutsch 门逼近 $n$ 比特幺正变换时需要 $2^{2^n}$ 个门.
\end{sol}
\end{document}